% vim: set spelllang=da:
\documentclass{article}

\usepackage[utf8]{inputenc}
\usepackage[danish]{babel}
\usepackage{float}
\usepackage{fancyhdr}
\usepackage{amsmath}
\usepackage{color}
\usepackage{listings}
\usepackage{graphicx}
\usepackage{pdfpages}
\usepackage{booktabs}
% \usepackage{enumitem}
\usepackage[a4paper, top = 1in, bottom = 1in, left=1in,right=1in]{geometry}

\title{HCI - Individuel Aflevering}
\author{Peter Heilbo Ratgen}
\date{\today}

\begin{document}
\maketitle
Gennem designsprintet afveg vi flere gange fra den gængse proces. Dette var med
varierende succes, og ikke nødvendigvis meningen.

\section{Fordele og ulemper i processen}

% Hvilke afvigselser fungerede, og hvilke fungerede ikke? Hvad var godt eller
% skidt, ved disse afvigselser? Ville jeg afvige igen i et nyt designsprint.

\paragraph{Mandag}

% Mandag
% Manglende How Might We - notee
Mandag manglede vi at lave How Might We Notes fra kommentarerne som vi fik fra
eksperterne. Hvis vi havde gjort dette kunne vi måske nemmere få sat nogle
problemstillinger vores Miro board. Disse problemstillinger kunne vi have brugt
senere til at lave et bedre og mere veludviklet produkt.

\paragraph{Tirsdag}

% Tirsdag
% Sketching processen var ikke særlig god, vi tegnede ikke nogle "grove" ideer,
% divide/swarm?
% vi har ikke hver især lavet en soloution sketch
  % dette bryder med princippet om at alle skal sketche
    % bare fordi der var en der var bedre end andre til at tegne.

Vores sketching process var lidt rodet, hvor vi ikke helt fik fulgt den
udstukkede linje. Vi fik startet med lightning demos dette gav et godt indblik i
hvilke andre tjenester der eksister ude i verden allerede nu, og hvordan disse
fungerer. Vi kiggede også på en side om Elgiganten der havde interessante
pointer omkring kategorisering.  Viaplay havde en god måde at sortere film på.
Dette gav os bedre input for at kunne forsætte med at skitsere videre i
processen. Dog var der nogle misforståelser omkring processen her, så der
forekom en sammenblanding af de sketches vi skulle lave efter vi skulle beslutte
om vi skulle "divide or swarm".
% Her kunne der have været rart hvis alle gruppemedlemmer havde læst op og
% forstået hvad processen gik ud på.

% TODO: fix omtale af der kun er en enkelt swarm, vi arbejdede på mange
% forskellige, deraf problemet
Vi fik ikke aftalt om vi skulle "divide or swarm" på forskellige punkter i vores
map, da endte vi med at vi alle fokuserede på forskellige elementer. Det havde
været bedre hvis vi i fællesskab decideret havde besluttet os for at "divide or
swarm" bestemte punkter i mappet, dette havde formentlig givet en mere
struktureret måde at arbejde på.

Dernæst optegnede vi hver især nogle grove skitser omkring at søge på film.
Dette var en proces der fungerede rigtig godt, dette der var særlig godt var at
alle blev hørt omkring deres idéer og tanker omkring siden layout og andre
interessante koncepter som folk i gruppen kom op med.

Efter vi havde sketchet og præsenteret vores idéer forsatte vi til Crazy 8's.
Dette var en øvelse der fungerede særlig godt til at generere nye og anderledes
idéer. Dette er da man har så kort til at tegne otte forskellige skitser at man
er nødt til at tegne præcis det der falder én ind. Dette var med til at generere
den måde man kan vælge streamingtjenester på, så dette var en meget udbytterig
øvelse. Vi diskuterede og forklarede herefter vores Crazy 8's og hvad vi tænkte
med hver tegning vi havde lavet, dette var med til at give en fælles forståelse
for hvad vi hver især gjorde sig af tanker. % måske uddyb lidt her

\paragraph{Onsdag}

\paragraph{Torsdag}

% Torsdag
% Lidt et problem at man alle ikke kunne arbejde på sketchen

\paragraph{Fredag}

% Hvad har jeg lært af at følge beskrivelsen dette? Hvor ville jeg afvige i et
% fremtidigt designsprint. Hvorfor dette?

% fordel ved at bruge papir in sketching processen

\section{Til fremtidige sprint}

\end{document}

