% vim: set spelllang=da:
\documentclass{article}

\usepackage[utf8]{inputenc}
\usepackage[danish]{babel}
\usepackage{float}
\usepackage{fancyhdr}
\usepackage{amsmath}
\usepackage{color}
\usepackage{listings}
\usepackage{graphicx}
\usepackage{pdfpages}
\usepackage{booktabs}
% \usepackage{enumitem}
\usepackage[a4paper, top = 1in, bottom = 1in, left=1in,right=1in]{geometry}

\title{HCI - Individuel Aflevering}
\author{Peter Heilbo Ratgen}
\date{\today}

\begin{document}
\maketitle
Gennem designsprintet afveg vi flere gange fra den gængse proces. Dette var med
varierende succes, og ikke nødvendigvis meningen.

\section{Fordele og ulemper i processen}

% Hvilke afvigselser fungerede, og hvilke fungerede ikke? Hvad var godt eller
% skidt, ved disse afvigselser? Ville jeg afvige igen i et nyt designsprint.

\paragraph{Mandag}

% Mandag
% Manglende How Might We - notee
Mandag manglede vi at lave How Might We Notes fra kommentarerne som vi fik fra
eksperterne. Hvis vi havde gjort dette kunne vi måske nemmere få sat nogle
problemstillinger vores Miro board. Disse problemstillinger kunne vi have brugt
senere til at lave et bedre og mere veludviklet produkt.

\paragraph{Tirsdag}

% Tirsdag
% Sketching processen var ikke særlig god, vi tegnede ikke nogle "grove" ideer,
% divide/swarm?
% vi har ikke hver især lavet en soloution sketch
  % dette bryder med princippet om at alle skal sketche
    % bare fordi der var en der var bedre end andre til at tegne.

Vores sketching process var lidt rodet, hvor vi ikke helt fik fulgt den
udstukkede linje. Vi fik startet med lightning demos dette gav et godt indblik i
hvilke andre tjenester der eksisterer ude i verden allerede nu, og hvordan disse
fungerer. Vi kiggede også på en side om Elgiganten der havde interessante
pointer omkring kategorisering.  Viaplay havde en god måde at sortere film på.
Dette gav os bedre input for at kunne forsætte med at skitsere videre i
processen. Dog var der nogle misforståelser omkring processen her, så der
forekom en sammenblanding af de sketches vi skulle lave efter vi skulle beslutte
om vi skulle "divide or swarm".
% Her kunne der have været rart hvis alle gruppemedlemmer havde læst op og
% forstået hvad processen gik ud på.

% TODO: fix omtale af der kun er en enkelt swarm, vi arbejdede på mange
% forskellige, deraf problemet
Vi fik ikke aftalt om vi skulle "divide or swarm" på forskellige punkter i vores
map, da endte vi med at vi alle fokuserede på forskellige elementer. Det havde
været bedre hvis vi i fællesskab decideret havde besluttet os for at "divide or
swarm" bestemte punkter i mappet, dette havde formentlig givet en mere
struktureret måde at arbejde på.

Dernæst optegnede vi hver især nogle grove skitser omkring at søge på film.
Dette var en proces der fungerede rigtig godt, dette der var særlig godt var at
alle blev hørt omkring deres idéer og tanker omkring siden layout og andre
interessante koncepter som folk i gruppen kom op med.

Efter vi havde sketchet og præsenteret vores idéer forsatte vi til Crazy 8's.
Dette var en øvelse der fungerede særlig godt til at generere nye og anderledes
idéer. Dette er da man har så kort til at tegne otte forskellige skitser at man
er nødt til at tegne præcis det der falder én ind. Dette var med til at generere
den måde man kan vælge streamingtjenester på, så dette var en meget udbytterig
øvelse. Vi diskuterede og forklarede herefter vores Crazy 8's og hvad vi tænkte
med hver tegning vi havde lavet, dette var med til at give en fælles forståelse
for hvad vi hver især gjorde sig af tanker. % måske uddyb lidt her

Vi med det samme til vores solution sketch. Problemet her var at vi alle ikke
fik lavet en solution sketch for hver vores idé. I stedet blev i fællesskab
enige hvilke elementer der skulle med i sketchen ud fra vores tegninger. Dette
underminerer dog tanken om at skulle sketche og at alle har noget at bidrage
med. Det havde været bedre hvis vi havde lavet en solution sketch hver, i stedet
for at der var ét enkelt gruppemedlem der sad og tegnede på solution sketchen.
Og den solution sketch vi fik lavet var også for pæn i forhold til hvad der
egentlig forventes af en solution sketch. Derudover kunne det have været godt
med flere forklaringer til den solution sketch der blev udarbejdet.

\paragraph{Onsdag}

Onsdags dotvoting blev i stedet gjort på vores skitser i stedet for solution
sketches. Hvis vi havde gjort dette på solution sketches havde det formentlig
været nemmere at lave et mere detaljeret storyboard. Det havde også givet bedre
mening siden der allerede ville være knyttet kommentarer til disse løsninger.

Storyboardet var en gode måde at visualisere brugerens flow på vores side.
Storyboardet gav et godt overblik, selvom det ikke var meget specifikt, hvad
angik det grafiske. Hertil kommer kommentarerne der var knyttet til
storyboardet, der gjorde det nemmere at sætte sig i brugeren sted og forstå
brugerens rejse igennem systemet, fra start til slut.


\paragraph{Torsdag}

% Torsdag
% Lidt et problem at man alle ikke kunne arbejde på sketchen
Torsdag omhandlede at bygge en prototype der kunne brugertestes. Her arbejdede
vi sammen i Figma. Her var det lidt problematisk at vi ikke alle kunne arbejde
sammen og måtte sidde i hold af tre med en computer for hvert hold i stedet for
at alle kunne arbejde på en side hver for hurtigere at kunne få udbygget en
bedre prototype.

Men det at få konkretiseret vores skitser og storyboard i Figma fungerede rigtig
godt. Specielt at kunne få interagere med prototypen på et meget tidligt stadie i
udviklingsfasen, således at man måske kunne skrotte dele af systemet uden at
skulle spilde noget egentlig udviklingsarbejde. Dermed er det lettere at iterere
på designet.

\paragraph{Fredag}
% hvad fungerede ved testning?
Fredag omhandlede testning af prototypen som vi havde bygget fredag. Processen
omkring testningen af holdenes forskellige prototyper fungerede godt, da der
efter hver test var tid til at diskutere det testende holds opdagelser i forhold
til at hvad de testede personer havde givet udtryk for. Her kunne interviewerne
også sparre for at fortælle hvad der var gået godt, samt hvilke formuleringer
der gav de mest anvendelige svar. Det var åbne spørgsmål som gav de bedste og
mest anvendelige svar.

Det fungerede også godt med at der var én interviewer og to der tog notater for
omkring hvad testpersonen gjorde og sagde.

\section{Fremtidige sprint}
% Hvad har jeg lært af at følge beskrivelsen dette? Hvor ville jeg afvige i et
% fremtidigt designsprint. Hvorfor dette?

% fordel ved at bruge papir in sketching processen
\paragraph{Sketching} At bruge papir i sketchingprocessen i stedet for digitale
værktøjer er noget jeg vil tage med i fremtidige designsprints. Det giver en
frihed at alle er på alle niveau, hvad angår værktøjer. I et fremtidigt
designsprint vil det lægge mig på sinde at der ikke er nogen der skal anvende
farvetuscher eller lignende, således at alle kan deltage på samme fod. For hvis
ikke alle deltager med de sammen forudsætninger hvad angår værktøj, vil man
underbevidst altid være disponeret for at vælge den der ser mest "færdig" ud.
this 

\end{document}

