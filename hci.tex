\documentclass{article}

\usepackage[utf8]{inputenc}
\usepackage[danish]{babel}
\usepackage{float}
\usepackage{fancyhdr}
\usepackage{amsmath}
\usepackage{color}
\usepackage{listings}
\usepackage{graphicx}
\usepackage{lastpage}
\usepackage{enumitem}
\usepackage[a4paper, top = 1in, bottom = 1in, left=1in,right=1in]{geometry}
\usepackage{tikz}
\usepackage{tikz-qtree}
\usepackage{listingsutf8}

\title{Human Computer Interaction}
\author{Peter Heilbo Ratgen}
\date{\today, 1. semester}


\begin{document}
\maketitle

\section{Undervisning 15. september}
\subsection{Opsamling på øvelsestimer} 
Sprint questions, skal mest dreje sig om projektet, og ikke så meget om
management spørgsmål som fx "hvordan kan vi finde funding?".
Kurset handler om at skabe ideer i den rigtige retning, vi skal ikke vide hvad
prprojektet ender som. Alle artefakter skal omhandle brugerens rejse.

\subsection{Sketching}
Er en visuel måde at brainstorme på. Det er ikke mening at det skal være pænt.
Pointen er at give et overblik. Alle folk kan sketche. En sketch er primært
firkanter, cirkler og streger og måske et ord. Et billede siger mere end 1000
ord. 
Det er vigtigt at tegne for at starte tankeprocesser. Det er også med til at
give et fælles overblik. Det er bedst at sketche på papir eller et whiteboard.
Pointen er at holde det basic med papir og blyant, lige så snart man involverer
teknologi, giver man en barrier-of-entry så en kunde kunne finde på ikke at
bidrage.

\subsection{Design sprint}
\paragraph{Lightning demos} Vi laver en liste af demo system kandidater. Vi
skriver 1-2 systermer og skriver dem på whiteboardet. Vi skal give hinanden en 3
minutters demo, hvor man starter med overall system og dykker ned i enkelt
særlige detaljer.

\begin{itemize}
  \item Noter
  \item Ideer
  \item Crazy 8s
    \subitem En god måde at give hjernen en løbetur på. Man deler papiret i 8 og
    laver 8 sketches på 8 minutter.
  \item Præsenter dine sketches
    \subitem Hver person præsenterer sine sketches. Vi vil gerne tale om vores
    ideer.
  \item Iterér 
    \subitem Nu har vi set nogle forskellige ting, så kan vi få nye ideer.
\end{itemize}

Bagefter trækker vi linjer mellem de forskellige komponenter. Således at vi kan
flytte rundt på komponenterne.

\end{document}
